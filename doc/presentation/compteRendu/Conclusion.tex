\section{Conclusion}

Notre application, SynthPro est entièrement fonctionnelle, fiable
et répond aux attentes du cahier des charges initial. Nous avons eu
le temps de créer des modules à la fois musicalement utiles et
intéressants à programmer.

\subsection{Ce qui s'est bien passé}

\begin{itemize}
\item
  L'organisation de l'équipe a été efficace et nous n'avons jamais eu
  à attendre qu'un membre ait à finir son travail pour pouvoir
  avancer. Il n'y a pas eu de situations de blocage~;
\item
  Les trois jours de conception se sont montrés efficaces et
  suffisants pour poser les bases du projet et permettre par la suite
  l'élaboration des premiers modules et leur communication~;
\item
  Notre planning a été respecté et nous n'avons souffert d'aucun
  retard~;
\item
  Le \emph{framework} Qt, de part sa richesse, nous a grandement
  facilité la tâche~:
  \begin{itemize}
  \item
    Extension du C++ grâce à ses structures de données, certaines
    macros et ses facilités de gestion de mémoire,
  \item
    Bibliothèque graphique complète et performante,
  \item
    Bibliothèque audio (QtMultimedia) simple à utiliser.
  \end{itemize}
\item
  L'utilisation de Git en système de versionnement nous a
  probablement fait gagner du temps par rapport à SVN, et nous
  n'avons pas eu de conflits majeurs.
\end{itemize}

\subsection{Problèmes rencontrés}

\begin{itemize}
\item
  Comme expliqué dans le PSM, le moteur audio a posé problème selon
  l'organisation de l'architecture. Utiliser la moins élégante était
  nécessaire pour le faire fonctionner. QtMultimedia étant très
  jeune, il y a peu d'aide en ligne~;
\item
  QtMultimedia ne dispose pas de gestion Midi~;
\item
  Nous avons dû utiliser l'héritage multiple, ce qui a rendu la
  gestion des classes plus complexe~;
\item
  Le système de Layout des QGraphicsView a été plus complexe à
  utiliser que prévu, l'aspect général des modules n'est pas
  totalement conforme à nos attentes~;
\item
  QtCreator, bien qu'assez pratique à utiliser, n'est pas le plus
  efficace des IDE, et ne possède que des options limitées au niveau
  refactoring. De plus, le compilateur QMake a tendance à ne pas
  recompiler certains fichiers, provoquant des erreurs de compilation
  qui ne disparaissent que si on recompile intégralement le projet.
\end{itemize}

\subsection{Acquisition}

\begin{itemize}
\item
  Nous avons tous consolidé nos connaissances en C++~;
\item
  Nous avons acquis une base de traitement du signal~;
\item
  Nous nous sommes rendu compte que passer du temps sur une bonne
  conception vaut largement les jours de code perdus à corriger les
  erreurs d'une mauvaise conception.
\end{itemize}

\subsection{Et si c'était à refaire ?}

\begin{itemize}
\item
  Peut-être utiliser les QWidgets plutôt que QGraphicsView.
  L'avantage de ces derniers était de meilleures performances, mais
  leur gestion est plus laborieuse~;
\item
  Passer plus de temps à faire la conception n'aurait pas été un mal,
  même si les refactorings multiples n'ont que très rarement empêché
  le code existant de fonctionner.
\end{itemize}

\subsection{Mot de la fin}

Nous sommes satisfaits de notre application et il n'y a, au final,
pas beaucoup de choses que nous aurions voulu faire qui n'ont pas
été intégrées. Nous espérons qu'elle vous apportera entière
satisfaction.

Les BackSynth Boys.

