\section{Modèle Métier}

\subsection{Le synthétiseur}

L'application que nous développons consiste à simuler numériquement
le comportement des synthétiseurs analogiques, dits à synthèse
soustractive, ainsi que l'aspect modulaire de leur utilisation.
Chaque synthétiseur est composé de \textbf{Modules} que l'on
pourra lier entre eux, tout en écoutant le résultat du
montage en temps-réel.

\subsection{L'interface}

L'application possède un environnement graphique qui permet à
l'utilisateur de visualiser l'intégralité des modules disponibles.
Il aura également à sa disposition une surface qu'il pourra remplir
à loisir en déposant des modules du type désiré. Les différents
paramètres des modules peuvent également être modifiables en
temps-réel.

\subsection{Les modules}

Le module est la partie atomique du montage. Il n'existe pas \emph{a
priori} de limite au nombre de modules que l'on peut créer. Un
module dispose d'au moins une entrée et/ou une sortie. Certains
modules, sources de flux, ne disposent pas d'entrée, comme les
oscillateurs dont le but est de générer des ondes. D'autres, appelés
\emph{puits}, n'ont pas de sortie et sont utiles en fin de
montage pour, par exemple, diriger le son vers une sortie audio ou
afficher le résultat à la manière d'un oscilloscope.

\subsection{Les entrées/sorties}

\subsubsection{Généralités}
\label{ports-virtuels}

Un module possède au minimum une entrée et/ou une sortie (appelés
également \textbf{ports}). La conception de base des synthétiseurs
préconisait l'utilisation de démultiplexeurs et multiplexeurs afin
de pouvoir, respectivement, multiplier une entrée en plusieurs
sorties pour leur faire subir un traitement différent, ou au
contraire mixer plusieurs entrées en une sortie.

Ce concept nous a paru quelque peu lourd et peu intuitif, et
l'avantage de la simulation est qu'elle permet de
ne pas se limiter au monde physique tel qu'il existe. Ainsi, nous
avons choisi de nous passer de ces étapes intermédiaires.

Notre concept est le suivant~: chaque entrée et sortie d'un module
se duplique à partir du moment où l'une d'elle est utilisée. En
conséquence, les modules apparaissent initialement dotés d'une
seule entrée et/ou sortie, mais il suffit de la connecter
pour en voir apparaître une nouvelle, libre. L'avantage est double~: cela simplifie le montage qui n'est plus pollué par des
multiplexeurs et démultiplexeurs, et chaque module ne dispose plus
que du strict nécessaire, simplifiant encore le montage.

\subsubsection{Les Gates}

Les ports peuvent éventuellement être de type \textbf{Gate} (\textbf{Gate
in} pour les entrées, \textbf{Gate out} pour les sorties).

Qu'ils soient de type Gate ou non n'apporte pas de contrainte
particulière, mais implique une sémantique légèrement différente
par rapport aux ports conventionnels. Les Gates sont utilisées
principalement pour réagir à un front montant ou descendant et
déclencher une action en conséquence. Ainsi, le Gate Out du module
Keyboard (clavier virtuel) produira un front montant à chaque
pression de touche et un front descendant à son relâchement.
Branché à un module ADSR, cela lui permet de détecter chaque
nouvelle note et déclencher la modulation du volume.

\subsubsection{Les \emph{buffers}}

Le simulateur code les signaux analogiques par échantillonnage à $44100$ Hz. Les modules
traitent ensuite ces valeurs par lots. Chaque port d'entrée ou de sortie possède un
\emph{buffer} contenant le lot de valeurs à traiter.

\subsection{Les câbles}

Pour relier un port à un autre, l'utilisateur peut créer des câbles
virtuels simplement en désignant le port source vers le port de
destination. La seule contrainte imposée est qu'un port d'entrée
doit forcément être relié à une sortie. Il est également tout à
fait possible de créer des boucles dans le montage en
«~nourissant~» l'entrée d'un module avec la sortie de ce même
module, ou la sortie d'un module suivant dans le montage.
