\section{Introduction}

Ce projet, baptisé \textbf{SynthPro}, consiste à réaliser d'une
application permettant de simuler numériquement les sons issus des
premiers synthétiseurs analogiques, dits à synthèse soustractive.

Le principe de l'application est basée sur la modularité :
l'utilisateur doit pouvoir assembler des modules sonores entre eux
et entendre le résultat du montage en temps-réel. Il n'y a,
\emph{a priori}, pas de limitation quant au nombre de modules
pouvant être ajoutés, et la manière dont ils sont connectés doit
rester aussi permissive que possible.

\subsection{L'équipe}

\subsubsection{Présentation de l'équipe}

Notre équipe, portant le nom de \textbf{BackSynth Boys} (en
l'honneur d'un certain \emph{boys band}) est composée de quatre
personnes, chacune ayant un rôle particulier à remplir dans la
gestion du projet en lui-même, mais bien sûr également dans la
conception et la production de l'application :

\begin{itemize}
\item
  Julien Richard-Foy : \textbf{responsable Conception}. Chargé de la
  bonne conception et structuration du projet. S'est occupé de la
  partie Métier, de la communication entre celle-ci et l'interface
  graphique, et l'interface graphique en elle-même ;
\item
  Maxime Simon : \textbf{reponsable Qualité}. Chargé de la bonne
  formation des programmes écrits, aussi bien structurellement que
  syntaxiquement, et de la pertinence des tests effectués. S'est
  également occupé de l'interface et de la communication avec la
  partie Métier ;
\item
  Julien Névo : \textbf{responsable Projet}. Chargé de faire les
  rapports d'avancement du travail auprès de M. Plouzeau. S'est
  occupé du moteur audio, de la génération sonore des VCO et VCF, et
  de divers modules ;
\item
  Cyrille Folliot : \textbf{responsable Documentation}. Chargé de la
  qualité de la documentation produite. S'est également attelé au
  développement des modules et de la partie Métier.
\end{itemize}
\subsubsection{Organisation du temps de travail}

Quatre semaines nous étaient imparties pour mener à bien ce projet.
Les trois premiers jours furent utilisés pour mettre en place
l'architecture du projet et la conception de la partie Métier. Une
fois cela fait, nous nous sommes répartis les rôles en fonction de
nos affinités pour le travail à effectuer. Le travail de chacun a
pu être fait de manière indépendante très rapidement.

Nous avons mis en place des \textbf{itérations}, très courtes, dans
lesquelles étaient consignées les tâches que chacun devait
accomplir. Le but étant qu'à la fin de chaque itération, le projet
fonctionne toujours et soit à chaque fois agrémenté de nouvelles
fonctionnalités. Une itération peut durer de 1 à 3 jours.

\subsubsection{Logiciels et outils de développement}

\paragraph{Outils de développement}

Notre projet est basé sur le \emph{framework} Qt, dans le langage
C++. Nous avons utilisé l'environnement fourni, QtCreator, qui
fonctionne aussi bien sous Linux et Windows. Notre utilisateur de
Mac~OS~X a utilisé Xcode.

\paragraph{Versionnement}

Nous avons choisi d'utiliser Git pour le versionnement de notre
application, et ce pour trois raisons :

\begin{itemize}
\item
  SVN est connu de tous, et nous désirions utiliser un nouveau
  logiciel pour les comparer~;
\item
  Git s'avère plus efficace à gérer les conflits entre les différents
  \emph{commits} que SVN. Ce projet étant mené par 4 personnes, il
  était nécessaire de prévoir ce genre de situations~;
\item
  GitHub, une interface \emph{Web} de Git et liée au dépôt de
  fichiers, fut très pratique pour gérer le projet. Elle permet
  notamment de voir l'historique de chaque fichier, du projet en
  général, et du travail effectué par chaque personne.
\end{itemize}

\paragraph{Documentation}

Les schémas UML ont été mis en oeuvre sous BouML. La plupart des
documents a été écrite dans le wiki de GitHub, puis convertie au
format LaTeX grâce à Pandoc.

